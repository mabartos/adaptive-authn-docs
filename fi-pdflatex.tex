%%%%%%%%%%%%%%%%%%%%%%%%%%%%%%%%%%%%%%%%%%%%%%%%%%%%%%%%%%%%%%%%%%%%
%% I, the copyright holder of this work, release this work into the
%% public domain. This applies worldwide. In some countries this may
%% not be legally possible; if so: I grant anyone the right to use
%% this work for any purpose, without any conditions, unless such
%% conditions are required by law.
%%%%%%%%%%%%%%%%%%%%%%%%%%%%%%%%%%%%%%%%%%%%%%%%%%%%%%%%%%%%%%%%%%%%

\documentclass[
  digital,     %% The `digital` option enables the default options for the
               %% digital version of a document. Replace with `printed`
               %% to enable the default options for the printed version
               %% of a document.
%%  color,       %% Uncomment these lines (by removing the %% at the
%%               %% beginning) to use color in the printed version of your
%%               %% document
  oneside,     %% The `oneside` option enables one-sided typesetting,
               %% which is preferred if you are only going to submit a
               %% digital version of your thesis. Replace with `twoside`
               %% for double-sided typesetting if you are planning to
               %% also print your thesis. For double-sided typesetting,
               %% use at least 120 g/m² paper to prevent show-through.
  nosansbold,  %% The `nosansbold` option prevents the use of the
               %% sans-serif type face for bold text. Replace with
               %% `sansbold` to use sans-serif type face for bold text.
  nocolorbold, %% The `nocolorbold` option disables the usage of the
               %% blue color for bold text, instead using black. Replace
               %% with `colorbold` to use blue for bold text.
  lof,         %% The `lof` option prints the List of Figures. Replace
               %% with `nolof` to hide the List of Figures.
  lot,         %% The `lot` option prints the List of Tables. Replace
               %% with `nolot` to hide the List of Tables.
]{fithesis4}
%% The following section sets up the locales used in the thesis.
\usepackage[resetfonts]{cmap} %% We need to load the T2A font encoding
\usepackage[T1,T2A]{fontenc}  %% to use the Cyrillic fonts with Russian texts.
\usepackage[
  main=english, %% By using `czech` or `slovak` as the main locale
                %% instead of `english`, you can typeset the thesis
                %% in either Czech or Slovak, respectively.
  english, german, russian, czech, slovak %% The additional keys allow
]{babel}        %% foreign texts to be typeset as follows:
%%
%%   \begin{otherlanguage}{german}  ... \end{otherlanguage}
%%   \begin{otherlanguage}{russian} ... \end{otherlanguage}
%%   \begin{otherlanguage}{czech}   ... \end{otherlanguage}
%%   \begin{otherlanguage}{slovak}  ... \end{otherlanguage}
%%
%% For non-Latin scripts, it may be necessary to load additional
%% fonts:
\usepackage{paratype}
\def\textrussian#1{{\usefont{T2A}{PTSerif-TLF}{m}{rm}#1}}
%%
%% The following section sets up the metadata of the thesis.
\thesissetup{
    date        = \the\year/\the\month/\the\day,
    university  = mu,
    faculty     = fi,
    type        = mgr,
    department  = Department of Computer Systems and Communications,
    author      = Bc. Martin Bartoš,
    gender      = m,
    advisor     = {prof. RNDr. Václav Matyáš, M.Sc., Ph.D.},
    title       = {Adaptive Authentication in Keycloak},
    TeXtitle    = {Adaptive Authentication in Keycloak},
    keywords    = {Keycloak, authentication, security, open-source, risk-based, adaptive, Red Hat},
    TeXkeywords = {Keycloak, authentication, security, open-source, risk-based, adaptive, Red Hat},
    abstract    = {%
      Keycloak is an open-source identity and access management system that provides robust security solutions for applications and services. It acts as a centralized authentication and authorization platform, enabling organizations to secure their resources while offering Single Sign-On (SSO) capabilities. Keycloak is widely used in various industries, including web applications, microservices, and APIs, to protect sensitive data and streamline user access control. 
      
      The thesis aims to enhance the security of authentication processes in Keycloak. Traditional static authentication mechanisms, such as username and password, are vulnerable to various security threats, including password breaches, phishing attacks, and unauthorized access. Adaptive authentication might help mitigate these issues. 
      
      Adaptive authentication aims to improve the security posture by dynamically adjusting the authentication requirements based on contextual factors. That means the system can adapt its authentication methods and policies in response to changing circumstances.
    },
    thanks      = {%
      I am deeply grateful to my supervisor, Václav Matyáš, for his guidance and support throughout this thesis. I would also like to thank Marek Posolda for his valuable insights, which greatly enriched this work. Their contributions have been instrumental in its completion.
    },
    bib         = example.bib,
    %% Remove the following line to use the JVS 2018 faculty logo.
    facultyLogo = fithesis-fi,
}
\usepackage{makeidx}      %% The `makeidx` package contains
\makeindex                %% helper commands for index typesetting.
%% These additional packages are used within the document:
\usepackage{paralist} %% Compact list environments
\usepackage{amsmath}  %% Mathematics
\usepackage{amsthm}
\usepackage{amsfonts}
\usepackage{url}      %% Hyperlinks
\usepackage{markdown} %% Lightweight markup
\usepackage{listings} %% Source code highlighting
\lstset{
  basicstyle      = \ttfamily,
  identifierstyle = \color{black},
  keywordstyle    = \color{blue},
  keywordstyle    = {[2]\color{cyan}},
  keywordstyle    = {[3]\color{olive}},
  stringstyle     = \color{teal},
  commentstyle    = \itshape\color{magenta},
  breaklines      = true,
}
\usepackage{floatrow} %% Putting captions above tables
\floatsetup[table]{capposition=top}
\usepackage[babel]{csquotes} %% Context-sensitive quotation marks
\begin{document}
%% The \chapter* command can be used to produce unnumbered chapters:
\chapter*{Introduction}
%% Unlike \chapter, \chapter* does not update the headings and does not
%% enter the chapter to the table of contents. I we want correct
%% headings and a table of contents entry, we must add them manually:
\markright{\textsc{Introduction}}
\addcontentsline{toc}{chapter}{Introduction}

Theses are rumoured to be \enquote{the capstones of education}, so
I decided to write one of my own. If all goes well, I will soon
have a diploma under my belt. Wish me luck!

\chapter{Terminology}
\shorthandoff{-}
\begin{markdown*}{%
  hybrid,
  definitionLists,
  footnotes,
  inlineFootnotes,
  hashEnumerators,
  fencedCode,
  citations,
  citationNbsps,
  pipeTables,
  tableCaptions,
}
\section{Adaptive Authentication}
TODO - what is Adaptive Authn + Risk-based

\section{Keycloak}
Keycloak

\section{Service Provider Interface}

\end{markdown*}


\shorthandon{-}

\chapter{Existing solutions}
\shorthandoff{-}
\begin{markdown*}{%
  hybrid,
  definitionLists,
  footnotes,
  inlineFootnotes,
  hashEnumerators,
  fencedCode,
  citations,
  citationNbsps,
  pipeTables,
  tableCaptions,
}

This specific part of the thesis is focused on the analysis of existing solutions provided by other vendors that are present on the market.
It is very beneficial to be aware of solutions for Adaptive Authentication as there are already some companies and institutions that are trying to solve the same problem as is touched on in this thesis.

The intention behind analyzing existing solutions is to get information from the other vendors in order not to reinvent the wheel.
In order to create an optimal solution for Adaptive authentication, it is advantageous to recognize common patterns in these solutions, aggregate the best parts, enhance them, and tailor them to the needs of Keycloak. 

In the following sections, there are some brief descriptions of products provided by the mentioned vendors.
The list of existing solutions and their ordering is not sorted by any particular keys but rather gathered as the most promising and interesting solutions. 


\section{Okta}
OKTA
\section{Something}
Something

\section{Common patterns}

\end{markdown*}

\chapter{Analysis}
\shorthandoff{-}
\shorthandoff{-}
\begin{markdown*}{%
  hybrid,
  definitionLists,
  footnotes,
  inlineFootnotes,
  hashEnumerators,
  fencedCode,
  citations,
  citationNbsps,
  pipeTables,
  tableCaptions,
}

\section{Necessary parts}
TODO -


\end{markdown*}
\shorthandon{-}


\chapter{Implementation}
\shorthandoff{-}
\begin{markdown*}{%
  hybrid,
  definitionLists,
  footnotes,
  inlineFootnotes,
  hashEnumerators,
  fencedCode,
  citations,
  citationNbsps,
  pipeTables,
  tableCaptions,
}

IMPLEMENTATION
\end{markdown*}
\shorthandon{-}

\chapter{Inserting the bibliography}
After linking a bibliography data\-base files to the document using
the \verb"\"\texttt{thesis\discretionary{-}{}{}setup\{bib\discretionary{=}{=}{=}%
\{\textit{file1},\textit{file2},\,\ldots\,\}\}} command, you can
start citing the entries. This is just dummy text
\parencite{borgman03} lightly sprinkled with citations
\parencite[p.~123]{greenberg98}. Several sources can be cited at
once: \cite{borgman03,greenberg98,thanh01}.
\citetitle{greenberg98} was written by \citeauthor{greenberg98} in
\citeyear{greenberg98}. We can also produce \textcite{greenberg98}%
\ or %% Let us define a compound command:
\def\citeauthoryear#1{(\textcite{#1},~\citeyear{#1})}%
\citeauthoryear{greenberg98}%
. The full bibliographic citation is:
\emph{\fullcite{greenberg98}}. We can easily insert a bibliographic
citation into the footnote\footfullcite{greenberg98}.

The \verb"\nocite" command will not generate any
output\nocite{muni}, but it will insert its arguments into
the bibliography. The \verb"\nocite{*}" command will insert all the
records in the bibliography database file into the bibliography.
Try uncommenting the command
%% \nocite{*}
and watch the bibliography section come apart at the seams.

When typesetting the document for the first time, citing a
\texttt{work} will expand to [\textbf{work}] and the
\verb"\printbibliography" command will produce no output. It is now
necessary to generate the bibliography by running \texttt{biber
\jobname.bcf} from the command line and then by typesetting the
document again twice. During the first run, the bibliography
section and the citations will be typeset, and in the second run,
the bibliography section will appear in the table of contents.

The \texttt{biber} command needs to be executed from within the
directory, where the \LaTeX\ source file is located. In Windows,
the command line can be opened in a directory by holding down the
\textsf{Shift} key and by clicking the right mouse button while
hovering the cursor over a directory.  Select the \textsf{Open
Command Window Here} option in the context menu that opens shortly
afterwards.

With online services -- such as Overleaf -- or when using an
automatic tool -- such as \LaTeX MK -- all commands are executed
automatically. When you omit the \verb"\printbibliography" command,
its location will be decided by the template.

  \printbibliography[heading=bibintoc] %% Print the bibliography.


\appendix %% Start the appendices.
\chapter{An appendix}
Here you can insert the appendices of your thesis.

\end{document}
