\chapter{Conclusion} \label{conclusion}

This thesis has comprehensively explored the concept of adaptive authentication, focusing on its implementation in Keycloak.
The need for adaptive authentication arises from the inadequacies of traditional methods in addressing the dynamic nature of cybersecurity threats.
By adjusting authentication requirements based on user contexts and behavior, adaptive authentication offers a robust solution for enhancing security.

A clear understanding of aspects and requirements has been gained after comprehensive research in the adaptive authentication area and a comparative analysis of existing solutions.
Keycloak has proven to be a very suitable platform for implementing adaptive authentication due to its flexible architecture and comprehensive features.
The extendability of the Keycloak capabilities was done almost entirely separately from the server, besides extending the administrator console, and provides the ability to continue with separate development.

The architecture for achieving the required aspects has been reconsidered a few times due to the simplification of the approach or as specific issues related to the design occurred.
The implementation chapter combined theoretical knowledge with practical steps.
It showed how to integrate adaptive authentication mechanisms into Keycloak, highlighting the importance of the role of a policy engine, risk evaluators, and the use of AI to improve security decisions. 

In conclusion, I believe this thesis might significantly contribute to the field of adaptive authentication by offering theoretical insights and practical implementation strategies using Keycloak.
During the work on the thesis, I gained a lot of knowledge and particular expertise in the area.
Such a comprehensive topic required a lot of planning and brainstorming, as during that, I used common techniques for software development and deepened my practical usage of them.

Future research could explore further advancements in AI integration and the development of more sophisticated adaptive algorithms to counter emerging cybersecurity threats.
I feel the Keycloak community might be interested in this approach as it has a certain potential.
I want to continue developing it and bring closer a brighter future for authentication and Keycloak.