\chapter{Specification}

The goal of this thesis is to enhance authentication processes in Keycloak.
Adaptive authentication could promote Keycloak to a better place on the market by providing such a feature. 
The Keycloak community could benefit from it as well, as many other developers might be interested in the feature and be more engaged in the development.

The primary goal of this thesis is not to implement the whole solution and get it merged into the Keycloak repository.
It is rather focused on providing an analysis of how it could be achieved, how it could work, and in what way it could be implemented.
The output of individual items could be described as Proof of Concept (PoC).

PoC might be defined as a realization of a certain idea, method, or principle in order to demonstrate its feasibility or viability.
It could also verify that some concept or theory has practical potential, so it might be very useful in this situation.\cite{spec-poc} 
\newline
\newline
The overall structure of the thesis might be divided into several points and described as follows:
\begin{enumerate}
    \item \textbf{Extendability} -- Provide suitable SPIs for the adaptive authentication components.
    \item \textbf{Basic implementation} -- PoC for a basic implementation of the adaptive authentication solution.
    \item \textbf{Policy engine} -- PoC for an approach leveraging authentication policies.
    \item \textbf{Artificial intelligence} -- PoC for an approach leveraging artificial intelligence. 
    \item \textbf{External integration} -- PoC for an approach leveraging third-party services.
\end{enumerate}

\newpage

\section{Extendability}
One of the aims of the thesis is to provide reasonable extendability of the approach.
Specifically, suitable SPIs should be created for the adaptive authentication components in order to have the ability to tailor them to the unique requirements of a user.
Keycloak offers a very elegant way to extend components of the application without much hassle.

Users should be able to provide a custom approach for obtaining risk(contextual) factors data in an easy way.
It means that users can create new or edit the default risk factor providers via the SPI concept described in Section \ref{keycloak-spi}.
For example, obtaining the IP address of the device trying to authenticate is supplied from a proxy or load balancer outside of the application.
Additionally, the evaluation of the risk factors should be flexible and customizable.
Users or organizations extending the approach should be able to provide their own algorithms for the risk score evaluation.

\section{Basic Implementation}
The other objective pursued in this thesis is to provide a PoC for the basic implementation of the adaptive authentication solution.
The approach should be able to be included in an authentication flow, gathering risk factors data, evaluating them, determining what level of risk the authentication engine is facing, and appropriately reacting to it.
Users should be able to accommodate the solution into the authentication process and see the present outcome.

The basic implementation should contain an approach on how to assign certain authentication sub-flows, or authentication executions, to the decision branches based on the risk level.
Users should be able to define multiple risk levels and customize ranges of risk scores associated with the levels.

Users might also be capable of setting certain decision weights to the data supplied by the risk factors mechanisms.
This means that they can decide how much a particular risk factor should affect the overall risk score.
It is also imperative to have the possibility to disable the evaluation of the risk score from particular risk factors.

\section{Policy Engine}
The next goal of the thesis is to create a PoC for authentication policy evaluations.
Authentication policy should behave in such a manner, that comprehensive creation of conditions should be supported and adequately evaluated.
Specifically, authentication policy can define the exact flow that would be executed when prior conditions are met.

For example, a particular authentication policy could require MFA when the device is located outside of the US.
In that case, a user needs to specify the initial conditions as evaluating the location and specifying concrete action that will proceed.

The authentication policy might have the possibility of getting risk factors data and determining the action based on their values.
The policy could also explicitly use the calculated risk score to decide the authentication attempt's continuity.

The authentication policy engine would gather and execute all the policies in a specific way.
The engine could support the configuration of its behavior and manage data about the associated policies.
Users should be able to replace the default authentication policy engine provider with a custom provider based on specific needs and requirements.

\section{Artificial Intelligence}\label{spec-ai}
The other objective pursued in this thesis is to provide a PoC for leveraging artificial intelligence (AI) in the adaptive authentication approach.
Nowadays, AI and machine learning are rapidly evolving with promising results, and the range of their applicability is more expansive than before.
That approach might be very suitable even for adaptive authentication, as it might be used for calculating risk scores based on data from risk factors.

Data from risk factors might be analyzed and processed by AI mechanisms, which might lead to more educated decisions.
The goal of leveraging artificial intelligence is to suggest an approach to what parts of the solution might benefit from using AI and run some experiments with AI engines.

\section{External Integration} \label{spec-external-integration}
The last goal of the thesis is to provide a PoC for integrating third-party services into the current solution.
Integration with external services, in some cases, might overlap with the previous goal in Section \ref{spec-ai} related to AI.
Specifically, AI engines are usually some sort of remote service with predefined API. 

Integration with external services would be very beneficial for obtaining data from risk factors, which might be remote.
However, one needs to be aware of possible issues with remote services and the overall authentication process execution.
The credibility of remote sources may differ and needs to be adequately investigated to determine whether it is suitable to include in the risk score evaluation.
Additionally, the latency for obtaining the data might be very high, and some sort of timeout should be set for gathering the risk factors data.

It leads to the specification of providing a particular risk factor data processing, as the retrieval of these data should be done asynchronously.
As mentioned above, the timeout for each particular set risk factor should be set and respected during the evaluation.
It may be configurable what to do in such situations -- either to not include the risk factor in the evaluation or just decrease the weight of the risk factor.