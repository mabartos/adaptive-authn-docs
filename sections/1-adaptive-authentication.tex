\chapter{Adaptive Authentication}
\section{Introduction}

Authentication incorporates two primary domains: user authentication and data authentication. 

User authentication assures the identity verification of an individual to grant access to a system or service.
It ensures that individuals are who they claim to be before allowing them access to resources.

Data authentication, on the other hand, assures data integrity and origin, ensuring it remains unaltered during transmission or storage, focusing solely on data content.

Adaptive authentication extends user authentication, as it is an advanced security mechanism designed to dynamically adjust the level of authentication based on various factors such as user behavior, context, and risk levels.
Unlike traditional authentication methods that rely on static credentials or fixed Multi-Factor Authentication (MFA) processes, adaptive authentication continuously evaluates the risk associated with each authentication attempt and adapts the authentication requirements accordingly.

At its core, adaptive authentication usually leverages real-time data analysis and contextual insights to make informed decisions about the appropriate level of authentication needed to verify a user's identity.
This dynamic approach allows organizations to strengthen security measures while minimizing disruption to user workflows.
Unlike static authentication methods, which apply the exact authentication requirements to all users and sessions, adaptive authentication systems continuously evaluate risk factors and adjust authentication measures accordingly.

By leveraging real-time data and analytical capabilities, adaptive authentication enhances security while also optimizing the user experience.\cite{intro-logintc}

Adaptive authentication is closely tied to the Identity Thread Detection and Response (ITDR) security discipline.
Certain aspects of adaptive authentication are fundamental concepts of the ITDR strategies aimed at identifying and responding to identity threats.
\newpage

\section{Benefits}
As was stated in the introduction of adaptive authentication, the main benefits behind its usability are mainly the dynamics.
As in the traditional approaches, the static MFA is very often used when assessing authentication policies with predefined exact rules that the user might comply with or the system's state is in. 

Adaptive authentication may provide the ability to have a more frictionless approach when a user is trying to prove their identity.
\cite{intro-silverfort}\cite{intro-descope}
\newline
\newline
The summarization of the more exciting benefits might be as follows:

\begin{itemize}
    \item \textbf{Strengthen security} -- when a user is trying to access sensitive or important resources, additional factors might be required.
    \item \textbf{Context--aware protection} -- takes into account contextual information, such as device information, location, IP address, and behavioral patterns.
    \item \textbf{Flexibility and extendability} -- by leveraging more contextual data, the policies and rules might better comply with customers' needs.
    \item \textbf{Better user experience} -- may provide more frictionless authentication as a user is not required to provide too many aspects of their identity.
    \item \textbf{Integration with AI/ML} -- artificial intelligence, or machine learning, might be used for processing contextual data, assessing risk levels, and making more knowledgeable decisions.
    \item \textbf{Alignment with zero trust model} -- Adaptive authentication, in most cases, aligns well with the zero trust security model due to its main principle to 'never trust, always verify'. \cite{intro-incognia}
\end{itemize}

\newpage
\section{Challenges}
While the benefits of adaptive authentication are clear and compelling, it is essential to acknowledge that this approach might face several challenges.
Despite its dynamic nature and improved user experience, adaptive authentication presents several challenges that organizations must address to realize its full potential.
\newline
\newline
The summarization of the biggest challenges might be as follows:

\begin{itemize}
    \item \textbf{Complexity} -- the implementation of the processing user behavior and evaluation of contextual data might be very complex and time-consuming.  
    \item \textbf{Accuracy of risk assessment} -- as the contextual data might not be reliable and correct in a particular context, the risk evaluation needs to be properly executed -- we need to prevent the count of false positives and false negatives. 
    \item \textbf{Privacy and data protection} -- as the contextual data might help to strengthen security, these data need to be collected and processed -- compliance with data protection regulations and users' preferences need to be assessed.
    \item \textbf{Potentially non-deterministic} -- the benefit of integration of AI/ML might also bring some challenges as we can achieve non-deterministic behavior in processing contextual data and assessing risk levels -- it might be challenging to predict possible outcomes, which also makes testing harder.\cite{intro-logintc}\cite{intro-silverfort}\cite{intro-descope}
    
\end{itemize}

\newpage
\section{Authentication Factors} \label{authentication-factors}
Authentication factors play a crucial role in verifying the identity of users and granting them access to digital systems, applications, and data.
In the realm of cybersecurity, authentication factors serve as the building blocks of authentication mechanisms, providing layers of security to protect against unauthorized access and mitigate the risk of security breaches.

Authentication factors exist to address the inherent limitations and vulnerabilities of traditional password-based authentication methods.
While passwords have long been the primary means of authenticating users, they are susceptible to various security threats, such as brute-force attacks, phishing, and password reuse.
Authentication factors enhance security by introducing additional layers of verification beyond mere knowledge of a password.\cite{auth-factors-aratek}

Adaptive authentication uses authentication factors to evaluate the potential risk that a user accessing a particular resource is fraudulent.
\newline
\newline
These factors use contextual data obtained by arbitrary approaches and can be divided into a few categories:

\begin{enumerate}
    \item \textbf{Knowledge factor} -- Something you know.
    \item \textbf{Possession factor} -- Something you have.
    \item \textbf{Inherence factor} -- Something you are.
    \item \textbf{Location factor} -- Somewhere you are.
    \item \textbf{Behavior factor} -- Something you do.
\end{enumerate}

The first three authentication factors \textit{Knowledge factor}, \textit{Possession factor}, and \textit{Inherence factor} are the well-known traditional factors used in many applications managing identity access.
The last two authentication factors \textit{Location factor}, and \textit{Behavior factor}, are extensions of traditional approaches and are extensively used for adaptive authentication.\cite{auth-factors-descope}  

\newpage
\subsection{Knowledge Factor (Something you know)}
Knowledge factors involve users providing specific information or data to access a secured system.
The most common type of knowledge-based authentication is through passwords or personal identification numbers (PINs), which are used to limit system access.

Typically, in generic applications or network logins, users need to provide both a username/email and its associated password or PIN to gain entry. It is important to note that solely providing a username or email does not serve as an authentication factor.
Instead, it functions as the user identification within the system.

Authentication occurs when the provided password or PIN verifies the associated username or email, confirming the identity of the user.

\subsection{Possession Factor (Something you have)}
Possession factors require users to possess a specific physical object before gaining access to the system.
Typically, possession factors are managed through a device known to belong to the correct user.
A typical process flow for possession-based authentication might look like the one described in the following lines.

The user registers an account with a password, and their phone number is recorded at the time of registration.
The user logs in to their account with the username and password.
When the user requests access to the system, a one-time password is generated and sent to the mobile phone number of the user.
The user enters the newly generated one-time password and gains access to the system.

One-time passwords can be generated by a device like the RSA SecurID, or they may be automatically generated and sent to the user's cellular device via SMS.
In either case, the correct user must possess the device that receives or generates the one-time password to access the system.

\newpage
\subsection{Inherence Factor (Something you are)}
Inherence factors authenticate access credentials based on unique user characteristics.
These include fingerprints, thumbprints, palm or handprints, as well as voice and facial recognition and retina or iris scans.

When systems proficiently identify users via their biometric data, inherence stands as one of the most secure authentication methods.
However, a downside is the potential loss of flexibility in account access.

For instance, a system requiring a fingerprint scan can only be accessed on devices equipped with the necessary hardware.
While this restriction enhances security, it may diminish user convenience. \cite{auth-factors-rublon}

\subsection{Location Factor (Somewhere you are)}
Location factors allow administrators to implement geolocation security checks to verify user location before granting access to an application, network, or system.

Consider a multinational corporation with offices in various cities worldwide.
In this scenario, a security analyst might identify a user attempting to access the network from an IP address originating from a country different from their assigned office.

Geolocation security ensures access is restricted to users within a specified geographic area.
While IP addresses offer insight into network traffic origin, hackers can avoid detection by using VPNs\footnote{Virtual Private Network} to mask their location.

Alternatively, MAC\footnote{Medium Access Control} addresses, unique to each computing device, can serve as a location-based authentication factor, allowing system access only from authorized devices within a defined range.

\newpage
\subsection{Behavior Factor (Something you do)}
Behavior factors depend on user actions to access the system. In systems accommodating behavior-based authentication factors, users may configure a password by executing specific actions within a defined interface, which they can subsequently replicate to verify their identity.

Consider keystroke dynamics and mouse movement patterns as examples of behavior-based authentication factors.
Keystroke dynamics can study the unique way a user types on a keyboard.
Mouse movement patterns can analyze the way a user moves the mouse while interacting with a system. \cite{auth-factors-aratek} \cite{auth-factors-sumologic} \cite{auth-factors-globalknowledge} 

\section{Risk Levels}
Risk levels in adaptive authentication typically refer to the likelihood that a particular access attempt is fraudulent or unauthorized.
These risk levels are often categorized into low, medium, and high, though the specific criteria for each level can vary depending on the implementation.

\textbf{Low-risk access} attempts might involve familiar devices, typical login times, and consistent behavior patterns. In these cases, the authentication process may be streamlined with minimal additional verification steps.

\textbf{Medium-risk access} attempts might involve slightly unusual behavior, such as logging in from a new location or using a different device.
In these cases, the system may prompt for additional verification, such as a one-time password sent via email or SMS.

\textbf{High-risk access} attempts might involve highly suspicious behavior, such as multiple failed login attempts, login from a known compromised device, or access from a location far from the user's usual ones.
In these cases, the system may require stricter authentication measures.

\shorthandon{-}