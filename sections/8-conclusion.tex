\chapter{Conclusion} \label{conclusion}

This thesis has comprehensively explored the concept of adaptive authentication, focusing on its implementation in Keycloak.
The need for adaptive authentication arises from the inadequacies of traditional methods in addressing the dynamic nature of cybersecurity threats.
By adjusting authentication requirements based on user contexts and behavior, adaptive authentication offers a robust solution for enhancing security.

A clear understanding of aspects and requirements has been gained after comprehensive research in the adaptive authentication area and a comparative analysis of existing solutions.
Keycloak has proven to be a very suitable platform for implementing adaptive authentication due to its flexible architecture and comprehensive features.
The extendability of the Keycloak capabilities was done almost entirely separately from the server, besides extending the administrator console with non-standard user interface components for the management of authentication policies.
Other than that, I was surprised there were no limitations concerning the extension of Keycloak capabilities.

The architecture for achieving the required aspects has been reconsidered a few times due to the simplification of the approach or as specific issues related to the design occurred.
To be more specific, during the implementation, I recognized providing accidental complexity\cite{conclusion-accidental-complexity}, as the design of components for risk-based authentication introduced several additional abilities that were not essential for solving the problem.
It consisted of added attributes, operations, or relations between components.
They did not bring any extra value, so they have been simplified for better manageability.

The implementation chapter showed how to smoothly integrate adaptive authentication mechanisms into Keycloak, highlighting the importance of the role of a policy engine, risk evaluators, and the use of AI to improve security decisions. 

In conclusion, I believe this thesis might significantly contribute to the field of adaptive authentication by offering many relevant conceptual insights and practical implementation strategies using Keycloak.
During the work on the thesis, I understood there might not be an optimal solution for a problem, but many with certain deficiencies need to be considered.
I have learned a lot about the Keycloak internal codebase and important authentication aspects.
Such a comprehensive topic required a lot of planning and brainstorming, as during that, I used common techniques for software development and deepened my practical usage of them.

Future research could explore further advancements in AI integration and the development of more sophisticated adaptive algorithms to counter emerging cybersecurity threats.
I feel the Keycloak community might be interested in this approach as it has a certain potential.
I want to continue developing it and bring closer a brighter future for adaptive authentication and Keycloak.